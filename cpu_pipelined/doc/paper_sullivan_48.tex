\documentclass[a4paper]{IEEEtran}


\setlength{\textheight}{24cm}

\setlength{\textwidth}{16cm}

\setlength{\oddsidemargin}{0.2cm}

\setlength{\topmargin}{-2cm}

\setlength{\columnsep}{2cm}
\usepackage{kbordermatrix}% http://www.hss.caltech.edu/~kcb/LaTeX.shtml
\newcommand{\noindex}{\hspace*{-0.8em}}
\parskip=5mm

%\usepackage{helvet}\begin{equation}
%\renewcommand{\familydefault}{\sfdefault}
\linespread{1.5}

\evensidemargin=0.3cm
\usepackage{fancyhdr}
\usepackage{stackengine}
\parindent=0mm
\usepackage[spanish,activeacute,es-tabla]{babel}
\usepackage[utf8]{inputenc}
\usepackage{graphicx}
\usepackage{float} % for [H] location specifier
\usepackage{multirow}
\usepackage{pgfplots}
\pgfplotsset{width=12cm,compat=1.9}
% packages for mathematics
\usepackage{amsmath}
\usepackage{lipsum}
%\pagestyle{fancy}
%\fancyhf{}
\usepackage{adjustbox}
\usepackage{amsmath}
\usepackage[document]{ragged2e}
\usepackage{amssymb}
%\rhead{}
%\fancyhead[LE,RO]{Tarea 4 - EARS}
%\fancyhead[RE,LO]{Isaac Núñez Araya}
% packages for pdf hyperlinks
\usepackage[pdftex,pdftitle={Isaac Nunez Araya - Examen 1},hidelinks]{hyperref}
% packages for drawing circuits
\usepackage[binary-units=true]{siunitx}
\usepackage[american,cuteinductors]{circuitikz}
\usepackage{tikz}
\usetikzlibrary{positioning, shapes.multipart, arrows, shadows, backgrounds, fit}

\sisetup{per-mode=symbol,per-symbol = p}
\newcommand\textbox[1]{%
	\parbox{.333\textwidth}{#1}%
}

\tikzset{
	bluebox/.style={
		draw,
		rectangle,
		minimum height=4.5cm,
		fill=blue!50!white,
		align=center,
		inner sep=2ex
	},
	whitebox/.style={
		draw,
		rectangle,
		minimum height=1cm,
		fill=white,
		align=center,
		inner sep=2ex
	},
	item/.style={
		draw,
		inner sep=1ex,
		fill=white 
	},
	matrix/.style={
		draw,
		fill=white,
		text centered,
		minimum height=1em,
		drop shadow
	}
}
\begin{document}
\newcommand{\titlepaper}{Proyecto I: Sullivan48}%(cambiar por nombre)}
\title{Proyecto I: Sullivan48}
\renewcommand\IEEEkeywordsname{Palabras clave}
% AUTHOR(S) INFORMATION

\author{
	\centering
	\begin{tabular}[htbp!]{r l} 
		Isaac~Núñez~Araya & Ernesto Ulate Ramírez\\
		isaacnez@outlook.com & ernesto.ulate@gmail.com
	\end{tabular} \\
	Área de Ingeniería en Computadores \\
	Instituto Tecnológico de Costa Rica
}
\maketitle
\IEEEdisplaynontitleabstractindextext
\IEEEpeerreviewmaketitle
\begin{abstract}
	En este proyecto se busca desarrollar una arquitectura la cual tenga como propósito el procesamiento de imágenes mediante convolución (filtrado) en 2D directamente en microarquitectura.
\end{abstract}
\begin{IEEEkeywords}
	Arquitectura, Convolución, Procesador vectorial
\end{IEEEkeywords}
\end{document}